\documentclass[a4paper]{article}

\usepackage{amsmath}
\usepackage{tikz}
\usetikzlibrary{positioning}
\usepackage{amssymb}
\usepackage{listings}
\usepackage{url}

\usepackage[xetex, colorlinks=true,
            linkcolor=blue, citecolor=blue,
            urlcolor=blue,breaklinks]{hyperref}

\usepackage[swedish]{babel}
\usepackage[babel]{csquotes}
\usepackage[margin=2.5cm]{geometry}

\usepackage{fontspec}
\usepackage{xunicode}

\title{\textbf{Operating systems \\
    Problem set I \\
    Uppsala University -- Spring 2016 \\}}
\author{Linnea Ingmar \\
  Elsa Rick} 
\date{\today}
\begin{document}
\maketitle

\section{Operating systems}
\begin{enumerate}
  \item \textit{What is the overall purpose of an operating system?} \\
    An important task of the operating system is to schedule user and system processes and handle commands from the user. When scheduling processes, the operating system can create the appearance that processes are run concurrently. It is the layer between the user space and the hardware.
  \item \textit{How can the operating system make several programs execute seemingly "at the same time"?} \\
    The operating system can swap which process is currently executed very quickly. This creates the appearance that the processes are run simultaneously.
\end{enumerate}

\section{Processes}
\begin{enumerate}
  \item \textit{What is the difference between a program and a process?} \\
    A program is a set of instructions which is in human readable format. It is a passive entity stored on secondary storage. In comparison, a process is an active entity that needs resources such as memory and CPU time. A process is a program loaded into memory, either executing or waiting.
  \item \textit{What does the acronym PCB stand for?} \\
    Process control block.
  \item \textit{Whats the purpose of the PCB?} \\
    The purpose is to store information needed to manage the process, such as the pid of the process. 
\end{enumerate}

\section{System calls}
\begin{enumerate}
  \item \textit{What is the overall purpose of the system call concept? } \\
    The purpose is to interact directly with the operating system. It provides an interface to the services made available by an operating system.
  \item \textit{Why are exceptions important when implementing system calls?} \\
    System calls are initiated using exceptions. 
  \item \textit{Why are interrupts important when implementing system calls?} \\
    System calls generate an interrupt, which invokes the attention of the operating system, so that the system call can be executed.
  \item \textit{Why can't we simply use ordinary function calls when implementing system calls?} \\
    A call to an ordinary function is a call to a piece of code in user space, and does not have the same acces rights as a system call. So when it is necessary to execute some tasks which must be performed by the operating system, a system call is needed.
  \item \textit{What is the difference between the system call interface and the actual system calls?} \\
    A system call interface is an abstraction for a given programming language, and acts as a link to the actual system calls. The caller need not know exactly how a system call is implemented, only how to use it via the interface and what the result is when making the system call.
  \item \textit{A common pattern when using system calls is for the caller to first allocate storage and then pass a pointer to this storage to the system call - why? } \\ 
    It gives the caller a choice in where to store the result of the system call.
  \item \textit{Some system calls are blocking, i.e., the caller will block until some condition is met. How can blocking system calls be implemented?} \\
   One way is to put the process that used the blocking system call in a queue, waiting for the condition to be met. Meanwhile, another process can be executed. When the previous mentioned condition is met, an interrupt occurs, and the waiting process and the current executing process are swapped, and the waiting process starts executing again. 
  
\end{enumerate}

\section{fork()}
\begin{enumerate}
  \item \textit{What is the purpose of the fork() system call?} \\
    The fork() system call creates a new process (a child process).
  \item \textit{How many times does fork() return?} \\
    Twice if a succesful fork(), else once. 
  \item \textit{What are the possible return values of fork()?} \\
    In the case of a successful creation of a new process, fork() returns the pid of the created process in the parent process, and in 0 in the child process. If fork() fails to create a new process, -1 is returned in the parent process.
  \item \textit{What actions must be taken by the kernel when executing the fork() system call?} \\
    The kernel creates a PCB and allocates memory for the child process. It then puts a copy of the parent process in the allocated memory, and schedules the parent and child processes to execute "concurrently".
  
\end{enumerate}

\section{exit()}
\begin{enumerate}
  \item \textit{What is the purpose of the exit() system call?} \\
    The purpose of exit() is to exit (terminate) a process.
  \item \textit{What actions must be taken by the kernel when exectuting the exit() system call?} \\
    The return value of exit() must be stored in the PCB since another process might need to read the return value. The kernel also deallocates the allocated memory of the process. 
  \item \textit{If an exit status is given to exit(), how is this information handled by the kernel?} \\
    The exit status is stored in the PCB.
  \item \textit{How is exit() related to the  "zombie" process state?} \\
    The kernel cannot deallocate the PCB until another process has read the return value of exit(). Until the return value of an exited process has been read, it is said to be a zombie process.
  
\end{enumerate}

\section{wait()}
\begin{enumerate}
  \item \textit{What is the purpose of the wait() system call?} \\
    The purpose of wait() is to read the return value of a child process.
  \item \textit{What are the possible return values of wait()?} \\
    The PID of the exited child process is returned.
  \item \textit{How can the exit status of a terminated process be retrieved using the wait() system call?} \\
    The exit status is written to the address pointed to by the argument given to wait().
  \item \textit{Hos is wait() related to the  "zombie" process state?} \\
    Since wait() reads the exit status of the exited process, it "de-zombifies" the zombie process.
\end{enumerate}

\section{exec()}
\begin{enumerate}
  \item \textit{What is the purpose of the exec() family of system calls?} \\
    The purpose of exec() is to replace the executing process with a new program.
  \item \textit{When calling a function or a system call, normally you will return back to the caller again. Is this true for the exec() family of system calls (motivate)?} \\
    The exec() functions only returns if an error occurs. Otherwise it replaces the old process with the new one, so there is no process to which to return a value.
  \item \textit{If a parent process creates a pipe, then calls fork() and the child calls exec - will the child be able to access the pipe?} \\
    Yes, the parent process and the child can communicate via the pipe.

\end{enumerate}

\section{Signals}
\begin{enumerate}
  \item \textit{What is the purpose of signals?} \\
    To communicate between signals.
  \item \textit{What is a signal handler?} \\
    A signal handler is a function that handles a specific type of signal. If no specific signal handler is registered, the default signal handler handles the signal.
  \item \textit{What are the limitations of signals?} \\
    A signal contains very little information.
\end{enumerate}

\section{Pipes}
\begin{enumerate}
  \item \textit{What is a pipe?}\\
    A pipe is a connection between two I/O-obejcts. 
  \item \textit{How are file descriptors used together with pipes?}\\
    A file descriptor identifies an I/O object in the kernel, and a pipe is a special kernel object that is basically just a pair of file descriptors. 
  \item \textit{How do we create a pipe?}\\
   Using the system call pipe(int arg[2]), where arg[2] is an int array with two elements.
  \item \textit{How can we make two processes share a pipe in order to communicate with each other?}\\
    Pipe first, then fork.
  \item \textit{How do we read data from a pipe?}\\
    Using the system call read().
    \begin{enumerate}
    \item \textit{What happens if we read from a pipe with no data?}\\
      read() blocks until there is data to read.
    \item \textit{What happens if we read from a pipe with no writers?}\\
      The pipe will see end-of-file, and return 0. There's nothing left to read.
    \end{enumerate}
  \item \textit{How do we write data to a pipe?}\\
    Using system call write().
    \begin{enumerate}
    \item \textit{What happens if we write to a full pipe?}\\
      write() blocks until a sufficient amount of data has been read from the pipe, and the write() can be completed.
    \item \textit{What happens if we write to pipe with no readers?}\\
      write() sends a signal SIGPIPE to the caller.
    \end{enumerate}
  \item \textit{Why is it a good practice to all ways close unused file descriptors?}\\
    It's important that the reader can detect the end-of-file, which it cannot if there are open writers. Therefore, it is important that the reader closes its unused "pipe write" end.  
    The opposite is true as well (the writer must close its "pipe read" end), since we need to be able to recieve the SIGPIPE signal in case there are no readers. 
    
\end{enumerate}

\section{dup2()}
\begin{enumerate}
  \item \textit{What is dup2() doing to file descriptors?} \\
    dup2() takes two file descriptions as arguments. It makes a copy of the first one, and stores the copy in the second one. 
  \item \textit{How can this be useful?} \\
    If you want to redirect for example input or output of a process. 
\end{enumerate}

\section{Hidden state}
\begin{enumerate} 
  \item \textit{ How is it possible for rand() to return different values on consequitve calls?} \\
    The value used to compute the return value is updated every time rand() is called. 
  \item \textit{A parent process calls srand() to seed the pseduo random generator (PRNG) and then uses fork() to create a number of child processes. Each child generates a sequence of "random" number by calling rand(). Can you make any predictions (motivate) about the sequences?} \\
    Assuming that the rand() is just a number of calculations, all children will generate the same number, since they have the same seed.      
  
\end{enumerate}

\section{Reentrant code}
\begin{enumerate}
\item \textit{What is meant by reentrant code?} \\
  Code that can safely be interrupted and then continue again, without the information having been corrupted or changed by another process. Reentrent code can be shared by several different processes concurrently.
\end{enumerate}





\end{document}
